\chapter{Mesh-to-mesh interpolation}
By changing the mesh, an adaptive algorithm also changes
the function space associated with the mesh. Therefore, adaptivity
necessitates an interpolation step between the previous and adapted
meshes so that the simulation may be timestepped forward.

The natural algorithm for interpolating from a donor mesh to a target
mesh, called consistent interpolation, is the interpolation derived
from the solution values of the donor mesh. For each node $n_{\cal{T}}
\in \cal{N}_{\cal{T}}$ in the target mesh $\cal{T}$, a containing
element $K_{\cal{D}}$ is identified in the donor mesh $\cal{D}$, and
the solution $q(\pmb{x})$ is evaluated at the physical location of the
target node $n_{\cal{T}}$. Such an element $K_{\cal{D}}$ may be
identified by an advancing front algorithm \citep{lohner1995} or by an
R-tree spatial indexing algorithm \citep{guttman1984}.

This algorithm is cheap to implement and is bounded for piecewise
linear basis functions. It is not conservative: the integral of the
interpolant on the target mesh will not in general be the same as the
integral of the field on the donor mesh. The method can be diffusive:
unless nodes in the new mesh exist at the maxima and minima of the
function on the old mesh, the maxima and minima may be lost. The
algorithm is also not defined for discontinuous meshes.

An alternative is to project from the donor mesh to the target mesh;
that is, to find the element of the target function space which is
closest in the $L_2$ norm to the function being interpolated. This is
referred to as Galerkin projection \citep{george1998}. Doing this
between arbitrarily unrelated unstructured meshes requires the
construction of a supermesh of the target and donor meshes; this may
be thought of as the mesh of intersections of the elements of the
target and donor meshes
\citep{farrell2009a,farrell2011conservative}. Galerkin projection is
conservative, minimally diffusive, and can be made bounded, but is
more expensive than consistent interpolation.
